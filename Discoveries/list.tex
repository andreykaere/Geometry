\documentclass[12pt]{extarticle}

\let\Overrightarrow\overrightarrow
\let\vecarrow\overrightarrow


%other%
\usepackage{graphicx}
\usepackage{float}
\usepackage[margin=0.6in]{geometry}
\usepackage{caption}
\usepackage{csquotes}
\usepackage[export]{adjustbox}
\usepackage{wrapfig}
\usepackage{setspace}
\usepackage{anyfontsize}
\usepackage{titlesec}
\titleformat{\section}{
	\normalfont\fontsize{20}{20}\bfseries}{\thesection}{1em}{}
\titleformat{\subsection}{
	\normalfont\fontsize{17}{20}\bfseries}{\thesubsection}{0.1em}{}
\usepackage{relsize}
%other%

\usepackage[mathscr]{euscript}

%%\newcommand{\F}{\Oldmathbfcal{F}}
%math%
\usepackage{amsthm}
\usepackage{amssymb}
\usepackage{amsmath}
\usepackage{mathtools}
%%\usepackage[cal = pxtx, scr = dutchcal]{mathalfa}



%\usepackage{unicode-math}
%\newtheorem*{}{\textup{Лемма}}
\newtheorem*{theorem}{\textup{Теорема}}
\newtheorem*{remark}{\textup{Комментарий}}
%\renewcommand\qedsymbol{$\blacksquare$}
%\usepackage{parskip}

\usepackage{pgfplots}
\usepgfplotslibrary{polar}
\usepgflibrary{shapes.geometric}
\usetikzlibrary{calc}


\renewenvironment{proof}
    {\noindent \textit{Доказательство.}\\
	\indent $\square$}
	{ $\blacksquare$\\ }

\newenvironment{solution}
	{\vspace{-4.3mm} \noindent\textbf{Решение.}}


\renewenvironment{remark}
    {\noindent\textbf{Коментарий}}

\usepackage{tikz}
   \usetikzlibrary{calc}

\newcommand{\arc}[0]{
   \tikz [baseline = (N.base), every node/.style={}] {
	  \node [inner sep = 0pt] (N){}; %{$#0$};
      \draw [line width = 0.8pt] plot [smooth, tension=1.3] coordinates {
         ($(N.north west) + (-1.5ex,0.6ex+0.4ex)$)
         ($(N.north)      + (-0.75ex,0+0.4ex)$)
         ($(N.north east) + (0ex,0.6ex+0.4ex)$)
      };
   }
}

\renewenvironment{rcases}
  {\left.\begin{aligned}}
  {\end{aligned}\right\rbrace}

\DeclarePairedDelimiter\abs{\lvert}{\rvert}
\DeclarePairedDelimiter\norm{\lVert}{\rVert}

%\newcounter{example}[section]
\newenvironment{example}[1]{\noindent \textbf{Пример #1.}}

\let\mathb\mathbb


\newcommand{\N}{\mathb{N}}
\newcommand{\Z}{\mathb{Z}}
\newcommand{\R}{\mathb{R}}
\newcommand{\F}{\mathbfcal{F}}
\renewcommand{\P}{\mathbfcal{P}}
\newcommand{\C}{\mathcal{C}}
\newcommand{\V}{\mathcal{V}}
%\newcommand*{\Z}{\mathbb{Z}}
%math%

%fonts%
%\usepackage[russian]{babel}
%\usepackage{polyglossia}
%\setdefaultlanguage[spelling=modern]{russian}
%\setotherlanguage{english}
%\setmainfont{CMU Serif}
%\setsansfont{CMU Sans Serif}
%\setmonofont{CMU Typewriter Text}  
%%\setmathfont{Latin Modern Math}
%\usepackage{mathrsfs}
%\DeclareMathAlphabet{\mathcal}{T1}{TX}{m}{n}

%\usepackage{unicode-math}


%\let\Oldmathbfcal=\mathbfcal
%\renewcommand{\mathbfcal}{\mumble\Oldmathbfcal}

%%\newcommand{\F}{\Oldmathbfcal{F}}
%%\newcommand{\F}{𝓕}

%\usepackage{fontspec}
%\setmathfont{Latin Modern Math}

\DeclareSymbolFont{symbols}{OMS}{cmsy}{m}{n}
\DeclareSymbolFont{bsymbols}{OMS}{cmsy}{b}{n}
\DeclareSymbolFontAlphabet{\mathcal}{symbols}
\DeclareSymbolFontAlphabet{\mathbfcal}{bsymbols}

%\setmathfont[range = {2131}]{AMS}
%\usepackage{amsfonts}
%\usepackage{dsfont}

\begin{document}

\textbf{Problem 1. (Concurrent circles)}
\textit{
Consider a triangle \(ABC\) which has inscibed circle \(\omega\) with 
center \(I\). 
Let \(H_a, H_b\) and \(H_c\) denote the feet
of altitudes from \(A\),\(B\) and \(C\) respectively. Let 
line \(AI\) meet \(\omega\) at зщште \(A_1\),
line \(BI\) meet \(\omega\) at point \(B_1\),
line \(CI\) meet \(\omega\) at point \(C_1\).
%\(A_1\) = \(AI \cap \omega\),
%\(B_1\) = \(BI \cap \omega\),
%\(C_1\) = \(CI \cap \omega\).
Let \(\Omega_a\) be a circumcircle of triangle \(AH_aA_1\). 
Similarly, we define \(\Omega_b\) and \(\Omega_c\).
Prove, that \(\Omega_a\), \(\Omega_b\) and \(\Omega_c\) are coaxial.\\
}

\textbf{Problem 2. (Ellipse's property)}
\textit{
Consider an ellipse \(\P\) with foci \(A\) and \(B\) and an arbitrary 
point \(S\) outside of \(\P\). Let \(C\) be an arbitrary point on 
\(\P\). Let \(D\) be point on \(\P\), such that \(\angle ASC = \angle BSD\) and \(C\), \(D\) belong the same half-plane of the line \(AB\). 
Then tangent lines at \(C\) and \(D\) to \(\P\) intersect on the 
angle bisecor of \(\angle ASB\).\\
}
 

%%

\textbf{Problem 3. (Poncelet porism property)}
%%Гипотеза. Рассмотрим \(n\)-угольник \(A_1A_2…A_n\), вписанный и описанный около двух коник \(\alpha_1\) и \(\alpha_2\). Рассмотрим все пересечения его диагоналей вида 
%%\(A_iA_{i+2}\) (по модулю \(n\)). Они образуют свой \(n\)-угольник 
%%\(B_1B_2…B_n\).
%%По теореме Понселе многоугольник \(A_1A_2…A_n\) можно вращать между 
%%\(\alpha_1\) и \(\alpha_2\). Тогда оказывается, что 
%%многоугольник \(B_1B_2…B_n\) 
%%будет вращаться между своих двух фиксированных коник. 
%%
%%P.S. Проверил только для n=4,5,6.
%%При n=4 и α1,α2 окружностях получаем известный факт: все диагонали четырёхугольников, описанных и вписанных около данных окружностей, проходят через одну точку.
%%
\textit{
Hypothesis. Consider a \(n\)-gon \(\mathcal{A}=A_1A_2…A_n\), inscribed in and 
circumscribed about two conics \(\C_1\) and \(\C_2\) respectively.
Consider now all intersection points of the diangonals 
\(A_iA_{i+2}\) and let \(A_{n+j} = A_j\) for \(j \geqslant 1\). 
They form a \(n\)-gon \(\mathcal{B}=B_1B_2…B_n\) for \(n \geqslant 5\).
By the Poncelet theorem, we can \textquote{rotate} polygon 
\(\mathcal{A}\) between conics \(\C_1\) and \(\C_2\). 
Then, as it turns out, polygon \(\mathcal{B}\) rotates between some 
fixed conics \(\V_1\) and \(\V_2\).\\
\indent \textbf{Note}. If \(n = 4\), then polygon \(\mathcal{B}\) 
degenerates into a point \(\mathscr{B}\). By this theorem we can conclude, 
that as long as quadrilateral \(\mathcal{A}\) rotates, 
point \(\mathscr{B}\) is fixed. We will get well-known fact, if we let 
conics \(\C_1\) and \(\C_2\) be a circles.\\
\indent P.S. It's checked for cases \(n = 4,5,6\).\\
}


\textbf{Problem 4. (Property of Pascal Theorem for 4 points)}
\textit{
Consider a cyclic quadrilateral \(ABCD\), which is inscribed in circle 
\(\omega\). Let \(\ell_a, \ell_b, \ell_c, \ell_d\) be the tangent 
lines to the \(\omega\) at points \(A,B,C,D\) respectively. 
Denote the intersection point of the lines \(\ell_b\) and \(\ell_c\) 
by \(F\), the intersection point of the lines \(\ell_a\) and \(\ell_d\) 
by \(H\). Let \(E\) be the common point of the lines \(AB\) and 
\(CD\), \(G\) the common point of the lines \(BD\) and \(AC\). 
Let \(I\) be an arbitrary point on \(\omega\). Lines \(IE, IF, IG, IH\) 
intersect \(\omega\) at points \(M,L,K,J\).\\
\((a)\) Prove, that the points \(E,F,G,H\) are collinear. Moreover, 
\((E,G;F,H)\) is harmonic. (It's well-known fact) \\
\((b)\) Let \(\ell\) be the line, passing through the points \(E,F,G,H\). 
Prove, that the tangent lines at \(K,M\) -- \(\ell_k, \ell_m\) 
to \(\omega\), the line \(\ell\) and the line \(JL\) are concurrent.\\
}


\textbf{Problem 5. (Property of the humpty point of triangle)}
\textit{
Let \(\ell\) be the perpendicular line from orthocenter of the triangle 
\(ABC\) onto the line \(AM\), where \(M\) is midpoint of the side \(BC\).
Let \(H_b\) and \(H_c\) denote the feet of the altitudes from \(B\) and 
\(C\) respectively. Then lines \(H_bH_c\), \(BC\) and \(\ell\) are 
concurrent.
}

\end{document}
